\documentclass[11pt]{article}
\usepackage[utf8]{inputenc}
\usepackage{graphicx}
\usepackage{amsmath}
\usepackage{amssymb}
\usepackage{breqn}
\usepackage{braket}
\usepackage{geometry}
 \geometry{
 a4paper,
 total={160mm,257mm},
 left=30mm,
 top=20mm,
 }
\usepackage{etoolbox}

\title{Rate Matrix Elements for the Single Double Well}

\begin{document}

\maketitle

\section{Whatever}

Our goal is to calculate the Markov-approximated rate matrix elements in
the DVR-basis to the lowest order, given by
%
\begin{equation}\label{rate_matrix_element}
    \Gamma_{\mu\nu}^{(2)} = \frac{\Delta_{\mu\nu}^2}{2}
    \int_{0}^{\infty}\text{d}\tau e^{-(q_\mu -q_\nu)^2 S(\tau)}
    \cos\Bigl[ (E_\mu-E_\nu)\tau +(q_\mu -q_\nu)^2 R(\tau)\Bigr],
\end{equation}
%
where $S(\tau)$ and $R(\tau)$ are the real and respectively imaginary part 
of the bath autocorrelation function, the $q_\mu$ are the eigenvalues of
the position operator in the DVR-basis, further $\Delta_{\mu\nu}$ and the 
$E_\mu$ are given by the off-diagonal and respectively diagonal matrix
elements of the Hamiltonian in the DVR-basis.
We have already seen that in the weak-damping approximation, the bath
autocorrelation function is
%
\begin{align}
    S(\tau)&= Y(\cos\Omega\tau-1)+A\tau\cos\Omega\tau+B\tau+C\sin\Omega\tau+\mathcal{O}(\Gamma^2)\\
    R(\tau)&= W\sin\Omega\tau+V\Bigl(1-\cos\Omega\tau-\frac{\Omega}{2}\tau\sin\Omega\tau\Bigr)+\mathcal{O}(\Gamma^2),
\end{align}
%
with the temperature-dependent coefficients being
%
\begin{align}
    Y&=-\frac{4g^2}{\Omega^2}\coth\frac{\Omega}{2T}, \ \ \ W=\frac{4g^2}{\Omega^2} \\
    A&=\Gamma\frac{g^2}{2\Omega^2}\coth\frac{\Omega}{2T}, \ \ \ B=\Gamma\frac{8g^2T}{\Omega^2} \\
    C&=-\Gamma\frac{2g^2}{\Omega^3}\frac{\Omega/T+2\sinh\Omega/T}{\cosh\Omega/T-1}, \ \ \ V=\Gamma\frac{4g^2}{\Omega^3}.
\end{align}
%
Inserting all of this into (\ref{rate_matrix_element}) and absorbing the 
$(q_\mu-q_\nu)^2$ into the temperature-dependent coefficients, e.g.
redefining $(q_\mu-q_\nu)^2Y\rightarrow Y$, and renaming $E_\mu-E_\nu=\Delta E_{\mu\nu}$
the rate matrix elements then read
%
\begin{equation}\label{rate_matrix_element_2}
\begin{split}
    \Gamma_{\mu\nu}^{(2)} =\frac{\Delta_{\mu\nu}^2}{2}
    \int_{0}^{\infty}\text{d}\tau &e^{-Y(\cos\Omega\tau-1)-A\tau\cos\Omega\tau-B\tau-C\sin\Omega\tau} \\
    & \times \cos\Bigl[\Delta E_{\mu\nu}\tau +W\sin\Omega\tau+V\Bigl(1-\cos\Omega\tau-\frac{\Omega}{2}\tau\sin\Omega\tau\Bigr)\Bigr].
\end{split}
\end{equation}
%
This integral converges if the condition $B>A$ is met, corresponding to
%
\begin{equation}
    \frac{4T}{\Omega}>\coth\frac{\Omega}{2T}.
\end{equation}
%
After using Euler's formula for the cosine and rarranging, we can rewrite 
(\ref{rate_matrix_element_2}) as
%
\begin{equation}\label{rate_matrix_element_2}
\begin{split}
    \Gamma_{\mu\nu}^{(2)} &=\frac{\Delta_{\mu\nu}^2}{2}e^Y\int_{0}^{\infty}
    \text{d}\tau e^{-B\tau}\text{Re}\Bigl[ e^{-Y\cos\Omega\tau+\text{i}W\sin\Omega\tau} \\
    &\times e^{-A\tau\cos\Omega\tau-\text{i}V\Omega\tau/2\sin\Omega\tau 
    -C\sin\Omega\tau+\text{i}V-\text{i}V\cos\Omega\tau}e^{\text{i}\Delta E_{\mu\nu}\tau}\Bigr].
\end{split}
\end{equation}
%
We first deal with the exponential containing the coefficients of $0$-th 
order in $\Gamma$. Employing the Jacobi-Anger expansion 
%
\begin{equation}
    e^{\text{i}zcos\theta} = \sum_{n=-\infty}^{+\infty}\text{i}^n J_n(z)e^{\text{i}n\theta},
\end{equation}
%
it is possible to rewrite the exponential in question as 
%
\begin{equation}
    e^{-Y\cos\Omega\tau+\text{i}W\sin\Omega\tau} = 
    \sum_{n=-\infty}^{+\infty}(-\text{i})^n J_n(u_0) e^{\text{i}n\Omega\tau +n\Omega/(2T)},
\end{equation}
%
with $J_n(z)$ the $n$-th Bessel function of the first kind and 
$u_0 = \sqrt{W^2 - Y^2}$. Notice that $(-\text{i})^n J_n(u_0) \in \mathbb{R}$ 
for all $n$.
For the second exponential containing the terms
linear in $\Gamma$, we keep with the weak-damping approximation and thus 
expand the exponential, only keeping first order terms:
%
\begin{equation}
\begin{split}
    &e^{-A\tau\cos\Omega\tau-\text{i}V\Omega\tau/2\sin\Omega\tau 
    -C\sin\Omega\tau+\text{i}V-\text{i}V\cos\Omega\tau} \\
    &\approx 1 -A\tau\cos\Omega\tau-\text{i}V\Omega\tau/2\sin\Omega\tau 
    -C\sin\Omega\tau+\text{i}V-\text{i}V\cos\Omega\tau.
\end{split}
\end{equation}
%
Consequently, we write
%
\begin{equation}
\begin{split}
    \Gamma_{\mu\nu}^{(2)} &=\frac{\Delta_{\mu\nu}^2}{2}e^Y \sum_{n=-\infty}^{+\infty}
    (-\text{i})^n J_n(u_0)e^{n\Omega/(2T)}\int_{0}^{\infty}\text{d}\tau
    \text{Re}\Big[ e^{\text{i}n\Omega\tau+\text{i}\Delta E_{\mu\nu}\tau} 
    -A\tau e^{\text{i}n\Omega\tau+\text{i}\Delta E_{\mu\nu}\tau}\cos{\Omega\tau} \\
    &-\text{i}V\frac{\Omega}{2}\tau e^{\text{i}n\Omega\tau+\text{i}\Delta E_{\mu\nu}\tau}\sin\Omega\tau
    -C e^{\text{i}n\Omega\tau+\text{i}\Delta E_{\mu\nu}\tau}\sin\Omega\tau 
    -\text{i}V e^{\text{i}n\Omega\tau+\text{i}\Delta E_{\mu\nu}\tau}\cos\Omega\tau \\
    &+ \text{i}Ve^{\text{i}n\Omega\tau+\text{i}\Delta E_{\mu\nu}\tau}\Big].
\end{split}
\end{equation}
%
It is now a simple matter to calculate the real part of the terms in
the square brackets and calculate the integral of the remaining expression, 
which then just contains products of trigonometric functions. Doing so
and summing over $n=-1,0,1$, discarding terms containing $2\Omega$, we arrive
at the final form of the approximated rate matrix elements,
%
\begin{equation}
\begin{split}
    \Gamma_{\mu\nu}^{(2)} =\frac{\Delta_{\mu\nu}^2}{2}e^Y \Biggl\{  
    &\text{i}J_{-1}(u_0)e^{n\Omega/(2T)}\Bigl[ \frac{B}{B^2+(\Delta E_{\mu\nu}-\Omega)^2} 
    - \Bigl( \frac{A}{2}+\frac{V\Omega}{4} \Bigr) \frac{B^2}{(B^2+\Delta E_{\mu\nu}^2)^2} \\ 
    &-\Bigl( \frac{C}{2}-\frac{V}{2} \Bigr) \frac{\Delta E_{\mu\nu}}{B^2+\Delta E_{\mu\nu}^2} 
    -V\frac{\Delta E_{\mu\nu}-\Omega}{B^2+(\Delta E_{\mu\nu}-\Omega)^2} \Bigr] \\
    &+J_0(u_0) \Bigl[ \frac{B}{B^2+\Delta E_{\mu\nu}^2} - \Bigl( \frac{A}{2}-\frac{V\Omega}{4} \Bigr)
    \frac{B^2-(\Delta E_{\mu\nu}-\Omega)^2}{(B^2+(\Delta E_{\mu\nu}-\Omega))^2} \\
    &- \Bigl( \frac{A}{2}+\frac{V\Omega}{4} \Bigr)
    \frac{B^2-(\Delta E_{\mu\nu}+\Omega)^2}{(B^2+(\Delta E_{\mu\nu}+\Omega))^2} 
    + \Bigl( \frac{C}{2}+\frac{V}{2} \Bigr) \frac{\Delta E_{\mu\nu}-\Omega}{B^2+(\Delta E_{\mu\nu}-\Omega)^2} \\
    &+ \Bigl( \frac{V}{2}-\frac{C}{2} \Bigr) \frac{\Delta E_{\mu\nu}+\Omega}{B^2+(\Delta E_{\mu\nu}+\Omega)^2} 
    -V\frac{\Delta E_{\mu\nu}}{B^2+\Delta E_{\mu\nu}^2} \Bigr] \\
    &-\text{i}J_1(u_0) \Bigl[ \frac{B}{B^2+(\Delta E_{\mu\nu}+\Omega)^2} 
    -\Bigl( \frac{A}{2}-\frac{V\Omega}{4} \Bigr)
    \frac{B^2-\Delta E_{\mu\nu}^2}{(B^2+\Delta E_{\mu\nu}^2)^2} \\ 
    &+ \Bigl( \frac{C}{2}+\frac{V}{2} \Bigr) \frac{\Delta E_{\mu\nu}}{B^2+\Delta E_{\mu\nu}^2} 
    -V\frac{\Delta E_{\mu\nu}+\Omega}{B^2+(\Delta E_{\mu\nu}+\Omega)^2}\Bigr]
    \Biggr\}.
\end{split}
\end{equation}
%


\end{document}