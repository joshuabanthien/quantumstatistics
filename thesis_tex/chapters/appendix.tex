We want to evaluate the integral 
%
\begin{align}\label{bath_corr_func_app}
\begin{split}
    Q (\tau)
    &= \frac{1}{\pi} \int_{0}^{\infty} \text{d}\omega \frac{J \left( \omega \right)}{\omega^2}
    \coth\frac{\hbar\omega\beta}{2} \Bigl[ \left( 1-\cos\omega\tau \right) + \text{i}\sin\omega\tau \Bigr] \\
    &= \frac{\alpha}{\pi} 
    \int_{-\infty}^{\infty} \text{d}\omega \frac{1}{\omega} \frac{\Omega^4}{\left(\Omega^2-\omega^2\right)^2
    + \left(\Gamma\omega\right)^2}
    \coth\frac{\hbar\omega\beta}{2} \Bigl[ \left( 1-\cos\omega\tau \right) + \text{i}\sin\omega\tau \Bigr] \\
    &= \frac{\alpha}{\pi} 
    \int_{-\infty}^{\infty} \text{d}\omega \frac{1}{\omega} \frac{\Omega^4}{\left(\Omega^2-\omega^2\right)^2
    + \left(\Gamma\omega\right)^2}
    \coth\frac{\hbar\omega\beta}{2} \Bigl[ \left( 1-e^{\text{i}\omega\tau} \right) + \text{i}e^{\text{i}\omega\tau} \Bigr].
\end{split}
\end{align}
%
via the use of the residue theorem. To this end, we continue the integrand analytically 
and examine it for poles. For clarity, we will consider the real part and the imaginary
part seperately. The real part of (\ref{bath_corr_func_app}) reads
%
\begin{equation}\label{bath_corr_func_real_part_app}
    S(\tau)=\int_{-\infty}^{\infty}\text{d}\omega F(\omega)
    =\frac{\alpha}{\pi}\int_{-\infty}^{\infty}\text{d}\omega
    \frac{\Omega^4}{\left(\Omega^2-\omega^2\right)^2+\left(\Gamma\omega\right)^2}
    \frac{1-e^{\text{i}\omega\tau}}{\omega}\coth\frac{\hbar\omega\beta}{2}
\end{equation}
%
The first factor has poles at the solutions of
%
\begin{equation}
    \left(\Omega^2-\omega^2\right)^2+\left(\Gamma\omega\right)^2=0.
\end{equation}
%
Solving this equation is done by simple algebra, revealing four simple poles at
%
\begin{equation}
    \omega=\pm\frac{\text{i} \Gamma}{2}\pm\sqrt{\frac{\Gamma^2}{4}+\Omega^2}
    =\pm\bar{\Omega}\pm\frac{\text{i}\Gamma}{2}.
\end{equation}
%
The second factor does not possess any poles, since
\begin{equation}
    \lim_{z\rightarrow0}\frac{1-e^{\text{i}z\tau}}{z} =-\text{i}\tau.
\end{equation}
%
The third factor has simple poles at the zeroes of the complex sine function, which are
at $z=\text{i}k\pi$ for $k\in\mathbb{Z}$. The poles are therefore at
%
\begin{equation}
    \frac{\hbar\omega\beta}{2}=\text{i}k\pi 
    \Leftrightarrow \omega=\text{i}\frac{2\pi k}{\hbar\beta},
\end{equation}
%
which are the bosonic Matsubara frequencies. Having identified all poles, we now have to 
choose a suitable integration contour. In this case, we will choose the half-annulus lying 
in the upper complex half-plane with inner radius $r$ and outer radius $R$ traversed in the
mathematically positive sense. In the limit $R\rightarrow\infty$ and $r\rightarrow0$,
this contour encircles the two poles of the first factor that lie in the upper half-plane
as well as all the poles of the hyperbolic cotangent for which $k>0$.
With this contour, our integral takes the form
%
\begin{equation}
    \int_{-\infty}^{\infty}\text{d}\omega F(\omega)=2\pi\text{i}\sum_{k}\text{Res}(F,\omega_k)
    -\lim_{r\rightarrow 0}\int_{\mathcal{C}_2}\text{d}\omega F(\omega),
\end{equation}
%
where $\mathcal{C}_2$ denotes the inner half-circle and $\{ \omega_k \}_{k\in I}$ is 
the set of all encircled poles. First, we will compute the terms coming from the 
integration over the inner half-circle:
%
\begin{align}
\begin{split}
    \lim_{r\rightarrow 0}\int_{\mathcal{C}_2}\text{d}\omega F(\omega)
    &= \frac{\text{i}\alpha}{\pi}\int_{\pi}^{0}\text{d}\phi\lim_{r\rightarrow 0}
    \frac{\Omega^4}{\left(\Omega^2-r^2e^{i2\phi}\right)^2+\left(\Gamma re^{i\phi}\right)^2}
    \left(1-e^{\text{i}re^{i\phi}\tau}\right)\coth\frac{\hbar re^{i\phi}\beta}{2} \\
    &= -\frac{\text{i}\alpha}{\pi}\frac{2\text{i}\tau}{\hbar\beta}\int_{\pi}^{0}\text{d}\phi \\
    &= -\frac{2\alpha}{\hbar\beta}\tau.
\end{split}
\end{align}
%
The residue at the simple pole $\omega_k$ is calculated through the formula
%
\begin{equation}
    \text{Res}(F,\omega_k)=\lim_{\omega\rightarrow\omega_k}(\omega-\omega_k)F(\omega).
\end{equation}
%
At the pole $\omega=\bar{\Omega}+\text{i}\Gamma/2$, this amount to 
%
\begin{equation}
    \text{Res}(F,\bar{\Omega}+\text{i}\Gamma/2)=-\frac{\text{i}\alpha}{\pi}
    \left[ \frac{\bar{\Omega}^2-\Gamma^2/4}{4\bar{\Omega}\Gamma}-\frac{\text{i}}{4} \right]
    \frac{\sinh\hbar\beta\bar{\Omega}-\text{i}\sin\hbar\beta\Gamma/2}{\cosh\hbar\beta\bar{\Omega}-\cos\hbar\beta\Gamma/2}
    (1-e^{\text{i}\bar{\Omega}\tau}e^{-\Gamma/2\tau})
\end{equation}
%
and conversely for $\omega=-\bar{\Omega}+\text{i}\Gamma/2$
%
\begin{equation}
    \text{Res}(F,-\bar{\Omega}+\text{i}\Gamma/2)=-\frac{\text{i}\alpha}{\pi}
    \left[ \frac{\bar{\Omega}^2-\Gamma^2/4}{4\bar{\Omega}\Gamma}+\frac{\text{i}}{4} \right]
    \frac{\sinh\hbar\beta\bar{\Omega}+\text{i}\sin\hbar\beta\Gamma/2}{\cosh\hbar\beta\bar{\Omega}-\cos\hbar\beta\Gamma/2}
    (1-e^{-\text{i}\bar{\Omega}\tau}e^{-\Gamma/2\tau}).
\end{equation}
%